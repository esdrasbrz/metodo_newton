\documentclass{article}

\usepackage[utf8]{inputenc}
\usepackage[T1]{fontenc}
\usepackage[portuguese]{babel}
\usepackage{amsmath}
\usepackage{graphicx}
\usepackage{bbm}

\usepackage{xcolor}
% Definindo novas cores
\definecolor{verde}{rgb}{0,0.5,0}
% Configurando layout para mostrar codigos C++
\usepackage{listings}
\lstset{
  language=C++,
  basicstyle=\ttfamily\small,
  keywordstyle=\color{blue},
  stringstyle=\color{verde},
  commentstyle=\color{red},
  extendedchars=true,
  showspaces=false,
  showstringspaces=false,
  numbers=left,
  numberstyle=\tiny,
  breaklines=true,
  backgroundcolor=\color{green!10},
  breakautoindent=true,
  captionpos=b,
  xleftmargin=0pt,
}


\author{Esdras R. Carmo}
\title{Aproximando Raizes de Funções de Duas Variáveis}
\date{\today}


\begin{document}
    \maketitle

    \section{Aproximação afim de função $\mathbbm{R}^2 \longrightarrow \mathbbm{R}$}
        \paragraph{}
        Se $f: \mathbbm{R}^2 \longrightarrow \mathbbm{R}$ é uma função diferenciável, sabemos que a transformação
        linear que aproxima $f$ em $(x_0, y_0)$ é dada por $T$ da seguinte forma:

        \begin{align*}
            T&: \mathbbm{R}^2 \longrightarrow \mathbbm{R}\\
            T(u) &= \left[\begin{matrix} \frac{\partial f}{\partial x} (x_0, y_0)&
                                         \frac{\partial f}{\partial y} (x_0, y_0)\end{matrix}\right] \cdot u \text{,  } \forall u \in \mathbbm{R}^2
        \end{align*}

        \paragraph{}
        Assim, tomando $u = (x - x_0, y - y_0)$, conseguimos uma aproximação afim (linear) $L(x, y)$ para a função $f$ da seguinte forma:

        \begin{align*}
            u &= (x - x_0, y - y_0)\\
            L(x, y) &= T(x - x_0, y - y_0) + f(x_0, y_0)
        \end{align*}

    \section{Método de Newton para Funções $\mathbbm{R}^2 \longrightarrow \mathbbm{R}$}
        \paragraph{}
        O método de Newton consiste em que, partindo de um ponto inicial $(x_1, y_1)$, escolhido arbitrariamente, pode-se iterar de forma a
        obter $(x_2, y_2)$ a partir de uma das raizes da aproximação afim no ponto $(x_1, y_1)$. Dessa forma, iterando em um número suficiente
        de vezes, aproximamos nosso ponto $(x_n, y_n)$ de uma das raízes da função $f$.
        \paragraph{}
        Portanto, a partir de $(x_{n-1}, y_{n-1})$, temos:
        
        \begin{align*}
            L(x_n, y_n) &= 0\\
            \left[\begin{matrix} \frac{\partial f}{\partial x} (x_{n-1}, y_{n-1})&\frac{\partial f}{\partial y} (x_{n-1}, y_{n-1})\end{matrix}\right] \cdot
            \left[\begin{matrix} x_n - x_{n-1} \\ y_n - y_{n-1} \end{matrix}\right] &= - f(x_{n-1}, y_{n-1})
        \end{align*}

        \paragraph{}
        Dessa forma, encontre as soluções para a equação de forma que, fixando uma das variáveis seja possível calcular computacionalmente a outra.

        \paragraph{}
        O método não irá funcionar caso em algum dos pontos, ambas as derivadas parciais se anulem, sendo impossível encontrar o zero da aproximação afim.
        Caso isso aconteça, deve-se alterar a escolha do ponto inicial $(x_1, y_1)$.
        
    \section{Implementação em Linguagem C}
        \paragraph{}
        Para a implementação desse método, é preciso determinar um $\delta$ como condição de parada, de modo que $\left| f(x_n, y_n) - 0 \right| \leq \delta$.
        Além disso, é preciso sempre checar se ambas as derivadas não se anulam.
        \paragraph{}
        Para facilitar a implementação, pode-se determinar funções em C para representar a função $f$ e suas derivadas. Por exemplo, se escolhermos $f(x, y) = x^3 + y^2 - 3$,
        podemos implementar da seguinte forma:


        \begin{lstlisting}
#include <stdio.h>
#include <math.h>
#define DELTA 0.000001

// funcao f
double f(double x, double y) {
    return pow(x, 3) + pow(y, 2) - 3;
}

// derivada de f com relacao a x
double fx(double x) {
    return 3 * pow(x, 2);
}

// derivada de f com relacao a y
double fy(double y) {
    return 2 * y;
}

// funcao principal
int main(void) {
    /* ... */

    /* Exemplo de chamada das funcoes */
    a = f(2.3, 5.1);
    b = fx(2.1);
    c = fy(-6.8);

    /* ... */
}
        \end{lstlisting}

\end{document}
