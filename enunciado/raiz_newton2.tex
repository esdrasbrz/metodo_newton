\documentclass{article}

\usepackage[utf8]{inputenc}
\usepackage[T1]{fontenc}
\usepackage[portuguese]{babel}
\usepackage{amsmath}
\usepackage{graphicx}
\usepackage{bbm}

\usepackage{xcolor}
% Definindo novas cores
\definecolor{verde}{rgb}{0,0.5,0}
% Configurando layout para mostrar codigos C++
\usepackage{listings}
\lstset{
  language=C++,
  basicstyle=\ttfamily\small,
  keywordstyle=\color{blue},
  stringstyle=\color{verde},
  commentstyle=\color{red},
  extendedchars=true,
  showspaces=false,
  showstringspaces=false,
  numbers=left,
  numberstyle=\tiny,
  breaklines=true,
  backgroundcolor=\color{green!10},
  breakautoindent=true,
  captionpos=b,
  xleftmargin=0pt,
}


\author{Esdras R. Carmo}
\title{Aproximando Raizes de Funções de Duas Variáveis}
\date{\today}


\begin{document}
    \maketitle

    \section{Aproximação afim de função $\mathbbm{R}^2 \longrightarrow \mathbbm{R}$}
        \paragraph{}
        Se $f: \mathbbm{R}^2 \longrightarrow \mathbbm{R}$ é uma função diferenciável, sabemos que a transformação
        linear que aproxima $f$ em $(x_0, y_0)$ é dada por $T$ da seguinte forma:

        \begin{align*}
            T&: \mathbbm{R}^2 \longrightarrow \mathbbm{R}\\
            T(u) &= \left[\begin{matrix} \frac{\partial f}{\partial x} (x_0, y_0)&
                                         \frac{\partial f}{\partial y} (x_0, y_0)\end{matrix}\right] \cdot u \text{,  } \forall u \in \mathbbm{R}^2
        \end{align*}

        \paragraph{}
        Assim, tomando $u = (x - x_0, y - y_0)$, conseguimos uma aproximação afim (linear) $L(x, y)$ para a função $f$ da seguinte forma:

        \begin{align*}
            u &= (x - x_0, y - y_0)\\
            L(x, y) &= T(x - x_0, y - y_0) + f(x_0, y_0)
        \end{align*}

        \subsection{Método para estimar a raiz}
            \paragraph{}
            O método sugerido para estimar a raíz da função $f(x)$ consiste em uma simplificação
            do método de Newton.
            \paragraph{}
            Supondo que a função tem uma raíz no intervalo $I = ]i, s[$, sendo
            $i$ e $s$ constantes conhecidas, podemos aproximar a raíz da seguinte forma:

            \begin{enumerate}
                \item Tomamos $m$ o ponto médio do intervalo, i. e., $m = \frac{i + s} {2}$;
                \item Assumindo uma função crescente em $I$, então verificamos os seguintes casos:
                \begin{itemize}
                    \item Se $f(m) > 0$, então tomamos $s' = m$ e atualizamos o nosso intervalo para
                    $I' = ]i, s'[$;

                    \item Se $f(m) < 0$, então tomamos $i' = m$ e atualizamos o nosso intervalo para
                    $I' = ]i', s[$;
                \end{itemize}
                \item Sendo $\delta$ uma constante conhecida que dirá a precisão buscada pelo método,
                verificamos se, no novo intervalo formado, $f(s) - f(i) \leq \delta$. Caso a desigualdade se
                observe, a raíz de $f(x)$ será $x = \frac{i + s} {2}$. Caso contrário, deve-se retornar ao item 1 com
                o novo intervalo.
            \end{enumerate}

        \subsection{Implementação em linguagem C}
            \paragraph{}
            Sabendo disso, é possível escrever um programa que calcule $\sqrt{2}$ utilizando um intervalo $I$ e um
            erro $\delta$ conhecidos.

            \paragraph{}
            Para esse caso, suponha inicialmente $I = ]1, 2[$, i. e., $i = 1$ e $s = 2$. Utilize também $\delta = 0.000001$.
            Para isso, pode-se definir uma constante com o valor de $\delta$ no início do programa da seguinte forma:
            \begin{lstlisting}
#include <stdio.h>
#define DELTA 0.000001

int main(void) {
    /* ... */
    // referenciando a constante definida no inicio
    if (x <= DELTA) { /* ... */ }
    /* ... */
}
            \end{lstlisting}

            \paragraph{}
            Lembre-se de utilizar apenas a biblioteca \textit{stdio.h} e apenas estruturas de laços e condicionais. Além disso,
            é aconselhável a utilização de variáveis tipo \textit{double} quando necessitar de ponto flutuante.


    \section{Implementação para $\sqrt{n}$}
        \paragraph{}
        Quando o seu programa funcionar para o caso $\sqrt{2}$, altere-o para calcular $\sqrt{n}$ sendo $n$ um número natural.
        Para isso, receba um inteiro da entrada e modifique os limites iniciais do intervalo $I$ conforme o inteiro dado, e não
        mais como uma constante conhecida no momento da programação.
\end{document}
